\documentclass[11pt,a4paper]{article}
\usepackage{url}

\begin{document}
\begin{center}
  Spring Semester 2018 \\ Iowa State University\\[3ex]
  {\large LING 410X - Language as Data}\\[3ex]
  \textbf{Assignment 6} \\ \textbf{Submission Deadline: 14 April 2018 end of the day}
\end{center}


\paragraph{Instructions:} This assignment consists of only one question which carries 10 marks. Upload your submission as a Rmarkdown report (in html or doc or pdf format) with the file name as your\_Lastname\_A6.html/pdf/docx. 

\section*{Question 1} 
Take any two books (.txt files or any other format you want) from gutenberg.org, follow whatever pre-processing you want. Using the wordcloud package in R, do the following:
\begin{itemize}
\item Create a word cloud for each of these files.
\item Create a commonality cloud for both the files together
\item Create a comparison cloud for both files together.
\end{itemize}

Note: I want the report to be like a report. Not just presenting lines and lines of R code. 
%word clouds, commonality clouds, comparison cloud


References:
\begin{itemize}
\item Documentation of wordcloud package: \\ \url{https://cran.r-project.org/web/packages/wordcloud/wordcloud.pdf}
\item Here is a  blogpost on this package, which gives you some useful information: \\ \url{http://blog.fellstat.com/?cat=11}
\item An old tutorial website on mining twitter with R that shows examples of commonality and comparison clouds : \\ \url{https://sites.google.com/site/miningtwitter} 
- please note: This tutorial is old and it should only be used as an overview.
\end{itemize}
\end{document}

A6 on visualization


