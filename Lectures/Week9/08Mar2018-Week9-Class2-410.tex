\documentclass{beamer}
\usepackage[utf8]{inputenc}
\usepackage{graphicx}
\author[Sowmya Vajjala]{Instructor: Sowmya Vajjala}

\title[LING 410X]{LING 410X: Language as Data}
\subtitle{Semester: Spring '18}

\date{8 March 2018}

\institute{Iowa State University, USA}
%%%%%%%%%%%%%%%%%%%%%%%%%%%

\begin{document}

\begin{frame}\titlepage
\end{frame}

\begin{frame}
\frametitle{Class Outline}
\begin{itemize}
\item Midterm feedback summary
\item Question from Tuesday's class
\item Discussion about your final project ideas
\item Time for Assignment 4 
\item Reminder: Assignment 4 is due on 10th March 2018.
\end{itemize}
\end{frame}

\begin{frame}
\frametitle{Mid-term feedback: summary -1}
\framesubtitle{Delta for the course (for me)}
\begin{itemize}
\item Meet in lab every time (Hmm, not in my control!)
\item More help sessions like on 2/23 (I can try for one more)
\item Windows computers and errors (Hmm!)
\item Spend more time explaining concepts, discuss specifics of some functions (Okay, I will do this more often)
\end{itemize}
%5-10min
\end{frame}

\begin{frame}
\frametitle{Mid-term feedback: summary -2}
\framesubtitle{Delta for the student (for you)}
\begin{itemize}
\item Practice more 
\item Participate more
\item Read the book/read more about R functions
\end{itemize}
\end{frame}

\begin{frame}
\frametitle{}
\Large Last class' question
\end{frame}

\begin{frame}
\frametitle{do.call}
\begin{itemize}
\item It takes two arguments - a function (any function), and a list. 
\item What it does is - it applies that function on each element of a list, and returns output in the form of the first function
\item Example 1:
list1  $<-$ list(11,12,13,14) 
and say: \\ \tiny
(a) do.call(cbind, list1)
\\ (b) cbind(cbind(list1[[1]]),cbind(list1[[2]]),cbind(list1[[3]]), cbind(list1[[4]]))
\\ \normalsize (a) and (b) do the same thing.
\pause
\normalsize \item It applies cbind on each element of the list, and binds all these cbind outputs together. 
\item This in essence becomes rbind(list1) for such a simple case.
\end{itemize}
\end{frame}

\begin{frame}
\frametitle{do.call}
\framesubtitle{The real use}
\begin{itemize}
\item Let us say we have a embedded list:
\\ list2 = list(list(1,2,3), list(4,5,6), list(7,8,9))

\item Using cbind here:
\scriptsize 
\\ (c) do.call(cbind,list2)
\\ (d) cbind(cbind(list2[[1]]),cbind(list2[[2]]),cbind(list2[[3]]))
\normalsize \item (c) and (d) do the same thing, and gives us a matrix:
\\ 1 4 7
\\2 5 8
\\3 6 9 
\end{itemize}
\end{frame}

\begin{frame}
\frametitle{Discussion about final projects}
Each person/team: please share with others what ideas you have for your final project addressing the following questions:
\begin{itemize}
\item what is the corpus you have in mind?
\item how big is it? (how many texts/examples etc)
\item Do you know how to get it in a format R can understand?
\item what are the questions you want to address with that dataset?
\item how will you address these questions?
\item how would you evaluate what you did?
\end{itemize}
\pause Remember: initial report is due on 7th April!
\end{frame}

\begin{frame}
\frametitle{}
\Large Assignment 4 practice \& discussion
\end{frame}

\begin{frame}[fragile]
\frametitle{Working with csv files that have text columns}
\begin{itemize}
\item What is a csv file? 
\item While reading in a csv file:
\begin{verbatim}
data_train <- read.csv("a4-train.csv",header = TRUE)
\end{verbatim}
What is header = TRUE mean? \pause
\item How do I access individual columns in this? \pause
\item So, how do I access only text column? \pause
\item Let us say I do something like: \\ 
$mega <- paste(mydata["text"], collapse=" ")$
What happens? How do I get answers for Q1 in A4? \pause
\item stringsAsFactors=FALSE argument in read.csv
\end{itemize}
\end{frame}

\begin{frame}
\frametitle{Remaining time}
\Large Assignment 4 practice \& any other questions you have
\end{frame}

\begin{frame}
\frametitle{}
\Large Happy Spring break!!
\\ \small Topic for the week after break: Topic Models.
\end{frame}

\end{document}
