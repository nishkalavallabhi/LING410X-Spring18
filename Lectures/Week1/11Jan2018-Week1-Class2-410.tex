\documentclass{beamer}
\usepackage[utf8]{inputenc}
\usepackage{graphicx}
\author[Sowmya Vajjala]{Instructor: Sowmya Vajjala}

\title[LING 410X]{LING 410X: Language as Data}
\subtitle{Semester: Spring '18}

\date{11 Jan 2018}

\institute{Iowa State University, USA}
%%%%%%%%%%%%%%%%%%%%%%%%%%%

\begin{document}

\begin{frame}\titlepage
\end{frame}

\begin{frame}
\frametitle{Class outline}
\begin{itemize}
%\item News/announcements
\item Familiarizing yourself with R
\item Doing R tutorial by installing "Swirl" library. 
\item Installing R and R Studio on your computers
\item Assignment 1 description
\end{itemize}
\end{frame}

\begin{frame}
\frametitle{R and RStudio}
\begin{itemize}
\item R is the actual programming language, which we use to write code and which does all the processing.
\item R studio is a "development environment", which is like an interface to R, making it more convenient to use.
\item Installing both is easy - you can find this online and choose according to your operating system (windows, macos, linux variants etc)
\item Links are in the next slide. 
\end{itemize}
\end{frame}

\begin{frame}
\frametitle{Installations}
\begin{itemize}
\item On lab computers, R and R studio are already installed. You can open Rstudio by going to Applications and clicking Rstudio.
\item For personal laptops: You have to first download R and then Rstudio. Choose R version 3.4 and above.
\begin{itemize}
\item download R: \url{https://cran.rstudio.com/}
\item download R Studio: \url{https://www.rstudio.com/products/rstudio/download/}
\end{itemize}
\end{itemize}
Note: R Studio uses and needs R. Install R first. 
\end{frame}

\begin{frame}
\frametitle{Swirl - R package}
\begin{itemize}
\item an R package is a collection of R programs which you can use while working with your own data in R. 
\item a collection of such packages is called a library. I will use package and library interchangeably. \pause
\item Swirl is a R package that contains various interactive learning tutorials on topics related to R.
\item We will be working with this for the first 2 weeks to learn the basic vocabulary and grammar of R language.
\end{itemize}
\end{frame}

\begin{frame}
\frametitle{Installing Swirl}
\begin{itemize}
\item Open R-studio, Go to Tools $->$ Install packages, and search for swirl in the prompt. Click install.
\item Alternatively, you can type install.packages("swirl") in the console of Rstudio (usually seen in bottom left panel). 
\item Doing one of these will install swirl library into your computer's R.
\item Note: if you are using lab computers, your installations and changes will be lost once you logout. 
\end{itemize}
\end{frame}

\begin{frame}
\frametitle{using Swirl}
\begin{itemize}
\item Once you finished the installation, enter library("swirl") in the console of R studio.
\item If you see an error, it means you did not install swirl successfully. Let me know. 
\item If you successfully installed swirl, entering the library(swirl) should show you a prompt like this: \\  "Hi! Type swirl() when you are ready to begin."
\item If you see that, you are ready to begin.
\item Let me walk you through for the first and second lesson, and you can continue doing other lessons after that.
\end{itemize}
\end{frame}

\begin{frame}
\frametitle{Swirl Tutorial}
There are several small lessons inside the basic R programming course in Swirl. I want you to do atleast 2--3 lessons today. Try to finish up to Lesson 5 or 6, whenever you get time, and the rest, we can learn as we go. You are welcome to do as many exercises as you want, ofcourse! 
\end{frame}

\begin{frame}
\frametitle{Assignment 1 Description}
\begin{itemize}
\item 2 questions, 10 marks. 
\item Deadline: 27 January, midnight
\item first question: literature review - nothing about R.
\item second question: using R to process strings - you should finish the swirl() tutorial first and also follow an additional tutorial document I prepared. We will have time today, and on next thursday to familiarize ourselves with R. 
\end{itemize}
\end{frame}

\begin{frame}
\frametitle{Resources to learn or get help with R}
\begin{itemize}
\item \url{https://www.r-project.org/help.html}
\item \url{https://www.r-bloggers.com}
\item R Programming for Data Science - free ebook by Roger Peng (available on leanpub.com, uploaded to Canvas)
\item Coursera R courses by Johns Hopkins University (focus is on statistical analysis, not text)
\item Discuss with classmates, talk to me during office hours. 
\item Google search with the error messages you get or with your general questions
\item for Linguistics students: Look for a R book by Haraald Baayen (again, focus is more on statistics for linguists)
\item Other, once you are comfortable with R: \url{https://www.tidytextmining.com/tidytext.html}
\end{itemize}
\end{frame}

\begin{frame}
\frametitle{}
\begin{center}
\Large Please fill up the questionnaire
\end{center}
\end{frame}

\end{document}



\begin{frame}
\frametitle{News/Announcements}
\begin{enumerate}
\item Saw this article today: "Machine-Learning Algorithm Identifies Tweets Sent Under the Influence of Alcohol" (\url{https://goo.gl/44sN3N}) \pause
\item For female students: Consider joining RLadiesAmes meetup group. It is a community of female R users at ISU and in Ames area. \\
\url{https://www.meetup.com/R-Ladies-Ames/}, \url{https://twitter.com/rladiesames}) 
\end{enumerate}
\end{frame}
