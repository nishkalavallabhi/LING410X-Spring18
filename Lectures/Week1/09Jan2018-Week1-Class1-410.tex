\documentclass{beamer}
\usepackage[utf8]{inputenc}

\hypersetup{
    colorlinks,%
    citecolor=blue,%
    filecolor=blue,%
    linkcolor=blue,%
    urlcolor=blue 
    %urlcolor=mygreylink     % can put red here to better visualize the links
}

\author[Sowmya Vajjala]{Instructor: Sowmya Vajjala}

\title[LING 410X]{LING 410X: Language as Data}
\subtitle{Semester: Spring '18}

\date{09 January 2018}

\institute{Iowa State University, USA}
%%%%%%%%%%%%%%%%%%%%%%%%%%%

\begin{document}

\begin{frame}\titlepage
\end{frame}

\begin{frame}
\frametitle{Class outline}
\begin{itemize}
\item Introductions %20 min
\item Motivation for the course %10 min
\item Course objectives and Pre-requisites %5 min
\item Course Logistics %10min
\item Syllabus %15min
\item Group activity %15min
\item Pre-course questionnaire %15 min
\end{itemize}
\end{frame}

\begin{frame}
\frametitle{}
\begin{center}
\Large Introductions
\end{center}
\end{frame}

\begin{frame}
\frametitle{About me}
\begin{enumerate}
\item In ISU as Asst. Professor since January 2016.
\item PhD in Computational Linguistics, 2015.
\item Teaching experience:
\begin{itemize}
\item Language as Data (this course, last year)
\item Language and Computers (LING 120), Statistical Natural Language Processing (LING 515)
\item Introductory courses on programming and NLP for applied linguistics students
\item Technical communication for undergrad engineering students
\item Graduate level topics seminars for computational linguistics students (2012-13)
\end{itemize}
\end{enumerate}
\end{frame}

\begin{frame}
\frametitle{About you?}
\begin{enumerate}
\item Name
\item What do you do in ISU?
\item What are your interests related to computational analysis of language?
\item Why did you enroll in this course?
\end{enumerate}
\end{frame}

\begin{frame}
\frametitle{}
\begin{center}
\Large Motivation for the course
\end{center}
\end{frame}

\begin{frame}
\frametitle{Why this course?}
\begin{enumerate}
\item There is a lot of data available everywhere now. Text is one form of such data.
\item We write comments on amazon.com, read news, write blog posts, use twitter - all these forms of internet usage create lots and lots of textual data everyday.
\item Knowing how to work with text and extract some kind of information from it is a valuable and industry relevant skill.
\item This is the main reason for the creation of this course.
\end{enumerate}
\end{frame}

\begin{frame}
\frametitle{Some context from the news}
\begin{itemize}
\item Literature
\begin{enumerate}
\item "data mining reveals the six basic emotional arcs of story telling" (\url{https://goo.gl/i1xWTu})
\item authors "claim they created an algorithm that identifies the literary elements that guarantee a book a spot on the besteller lists." (\url{https://goo.gl/Hsjfmjl})
\end{enumerate}
\item Political science:
\begin{enumerate}
\item on linguistic analysis of debate transcripts from recent elections (\url{https://goo.gl/BNxTaa}, \url{https://goo.gl/7KIyWj})
\item 9/11 anniversary speeches: what next analysis tells us (\url{https://goo.gl/dCj477})
\end{enumerate}
\item Clinical psychology: "can you detect a manic episode on Twitter?" (\url{https://goo.gl/i5V7ST})
\end{itemize}
\small ... and so on. Not all these are super successful or anything. These are just a few examples to show the relevance of what you will learn in areas other than technology and computer science. 
\end{frame}

\begin{frame}
\frametitle{Why did I choose R?}
\begin{enumerate}
\item R is fastly becoming a popular language for data science and statistical analysis
\item R has a lot of support for creating visualization tools
\item Finally, you don't have to immerse yourself into programming to be able to write R code for your work.
(my personal opinion) \pause
\item So, I believe it is a suitable language to teach about doing text analysis to non-CS students.
\item What can CS students benefit: knowledge about text processing and R, programs about analysing textual data. 
\end{enumerate}
\end{frame}

\begin{frame}
\frametitle{}
\begin{center}
\Large Course Objectives and Pre-requisites
\end{center}
\end{frame}

\begin{frame}
\frametitle{Goals for the course}
\begin{enumerate}
\item Teach you basic methods and techniques of text processing
\item Teach you how to use R to analyse your own data
\item Teach how to create visualizations of text data  
\item Make you a comfortable R user who can search for and utilize existing R libraries to find solutions to your text processing problems
\item Make you work on a practical project that is relevant to the outside world
\end{enumerate}
\end{frame}

\begin{frame}
\frametitle{What are not the goals for this course}
\begin{enumerate}
\item Make you an expert programmer
\item Make you an expert R programmer
\item Make you a statistical analysis expert
\end{enumerate}
\end{frame}

\begin{frame}
\frametitle{Pre-requisites}
None. General curiosity about language, and a willingness to work with computer programs and tools.
\end{frame}

\begin{frame}
\frametitle{}
\begin{center}
\Large Course Logistics
\end{center}
\end{frame}

\begin{frame}
\frametitle{Meeting and Location}
\begin{itemize}
\item  Curtiss 0225 on Tuesdays, and Ross 0137 (Lab) on Thursdays, 9:30-10:50 am
\\ (Note that the Thursday classroom is different from what is put up on class scheduler.)
\item \textit{Office hours:} Tuesdays and thursdays, 11 am-12 noon (please email beforehand if there are specific issues to discuss. If this time does not work for you, send an email, and we can meet at a convenient time)
\item course website: on Canvas.
\item Credits: 3 
\end{itemize}
\end{frame}

\begin{frame}
\frametitle{}
\begin{center}
Format and Grading
\end{center}
\end{frame}

\begin{frame}
\frametitle{Course Format}
\begin{itemize}\itemsep2ex
\item weekly lectures and practical sessions
\item 6 assignments (70\%) + 1 final project (25\%)
\item 5\% for classroom participation/discussion participation
\end{itemize}
\end{frame}

\begin{frame}
\frametitle{Assignment and project deadlines}
\begin{itemize}
\item Assignment 1: 27 Jan 2018 - 10 marks
\item Assignment 2: 10 Feb 2018 - 10 marks
\item Assignment 3: 24 Feb 2018 - 10 marks
\item Assignment 4: 10 Mar 2018 - 15 marks
\item Assignment 5: 31 March 2018 - 15 marks
\item Assignment 6: 14 April 2018 - 10 marks
\item Group project: (25 Marks total)
\begin{itemize}
\item Initial report due: 7 April 2018 (5 marks)
\item Project presentation: 24-26 April (5 marks)
\item Project report, and code submission: Finals week, 3rd May (15 marks)
\end{itemize}
\end{itemize}
(3 assginments are already uploaded. Rest will be up in 2--3 weeks)
\end{frame}

\begin{frame}
\frametitle{Some general rules:}
\begin{itemize}
\item attendance: 80\% attendance requirement. Attendance is counted through per-class questions asked in the class, which can be answered in the discussion forum. 
\item missing a deadline is okay, but you will not get full credit.
\item long absence due to illness etc: please inform and follow university procedures.
\item cheating and plagiarism: see the course handbook, and university policy against plagiarism.
\item classroom behavior: please be punctual and do not do personal work in the class.
\item Disability accomodation: Please speak to Disability Resources Office (DRO) to officially request an accomodation.
\end{itemize}
\end{frame}

\begin{frame}
\frametitle{Other Issues}
\begin{itemize}
\item validating enrollment: who is enrolled? who is just here?
\item feedback about the course: 
\begin{enumerate}
\item Talk to me directly, or leave anonymous feedback at: \url{https://goo.gl/forms/9o4AmL9bpOfsHlRF2} or leave a paper feedback in my mailbox. 
\item Be confident enough to confront me and talk to me if there is a concern. 
\end{enumerate}
\end{itemize}
\end{frame}

\begin{frame}
\frametitle{}
\begin{center}
Syllabus
\end{center}
\end{frame}

\begin{frame}
\frametitle{Topics} 
\begin{itemize}
\item Introduction to the course, R, and linguistic analysis
\item Corpus preparation: methods to select, process and clean textual data
\item Keyword and Key-phrase extraction methods
\item text classification methods and their applications
\item topic modeling and its applications
\item methods of visualizing textual information
\end{itemize}
\end{frame}

\begin{frame} 
\frametitle{Text Book}
\begin{enumerate}
\item Primary textbook: "Text analysis for students of literature" by Matthew Jockers
\begin{itemize}
\item It is freely available as pdf from university network (from the publisher)
\item I will be using several other free online tutorials and stuff - urls will be given in appropriate locations
\item Software: R, RStudio (a graphical interface for R), and several text processing libraries in R (will talk about them as needed).
\end{itemize}
\end{enumerate}
\end{frame}

\begin{frame}
\frametitle{Any questions so far?}
\end{frame}

\begin{frame}
\frametitle{Next Class ..} 
\begin{itemize}
\item To do before next class:
\begin{enumerate}
\item Read the syllabus handbook carefully
\item If you have your own laptop, get that for thursday's class to install required stuff
\item Please note: Thursday's class is in ROSS 0137
\end{enumerate}
\item Next class: 
\begin{enumerate}
\item Installing R, Rstudio 
\item Working with R - tutorial
\item Assignment 1 description
\end{enumerate} 
\end{itemize}
\end{frame}


\begin{frame}
\frametitle{}
\begin{center}
\Large Group Activity -1 
\end{center}
Form into groups of 3 (know your classmates!) and figure out the answer for the given word sentiment classification problem. 
\end{frame}

\begin{frame}
\frametitle{Answers? -1}
Here is how I grouped the words: molistic, slatty, blitty, weasy, sloshful - perhaps belong to one group.
strungy, struffy, danty, cloovy, cluvious, brastic, frumsy - perhaps belong to one group. If so, then, answer to first question will be C and second question will be D. 
\\ source: NACLO 2007 puzzles. (\url{http://nacloweb.org/resources/problems/2007/N2007-AS.pdf})
\end{frame}

\begin{frame}
\frametitle{}
\begin{center}
\Large Group Activity - 2
\end{center}
Form into groups of 3 (know your classmates!) and figure out the answer for the given search relevance problem. 
\end{frame}

\begin{frame}
\frametitle{Answers? -2}
solution: \url{http://nacloweb.org/resources/problems/2007/N2007-BS.pdf}
\end{frame}

\end{document}


