\documentclass{beamer}
\usepackage[utf8]{inputenc}
\usepackage{graphicx}
\author[Sowmya Vajjala]{Instructor: Sowmya Vajjala}

\title[LING 410X]{LING 410X: Language as Data}
\subtitle{Semester: Spring '18}

\date{23 January 2018}

\institute{Iowa State University, USA}
%%%%%%%%%%%%%%%%%%%%%%%%%%%

\begin{document}

\begin{frame}\titlepage
\end{frame}

%Overview of last week - 10 min
%questions from end of last week -solution and discussion - 10 min
%SetWD - discussion - 5 min
%Discussion of my tutorial - issues I noticed, What you would face in real life - 10 min
%Summary of other R functions we saw - questions. - 5min

\begin{frame}
\frametitle{Outline}
\begin{itemize}
\item Review of last week
\item Some advice on coping with R, doing automated text analysis
\item What is a corpus?
\item Reading different formats of data
\end{itemize}
\end{frame}

\begin{frame}
\frametitle{}
\centering Review of last week
\end{frame}

\begin{frame}
\frametitle{Recap}
\begin{itemize}
\item Getting the most frequent words in a text
\item Looking for regular expression based patterns in text
\item grep/grepl functions - looking for occurrences of a given pattern/regular expression
\item sub/gsub functions - for substituting one pattern with another
\item stringr library, str\_count - function.
\end{itemize}  
\end{frame}

\begin{frame}
\frametitle{Last class' exercise}
\begin{itemize}
\item Alter 3 lines such that the program works with most of Gutenberg.org projects
\item I first identified one pattern (startsWith (*** START OF THE PROJECT)) \pause
\item I realized after class that it is inconsistent, and does not occur like that in all texts.
\item So, I changed to something that seemed more consistent.
\end{itemize}  
\end{frame}

\begin{frame}[fragile]
\frametitle{Original code: Specific to that particular text} 
\footnotesize
\begin{verbatim}
english <- scan("DollsHouse-Eng.txt", what = "character", sep = "\n")
english.start <- which(english == "DRAMATIS PERSONAE")
english.end <- which(english == 
      "(The sound of a door shutting is heard from below.)")
actual_english <- english[english.start:english.end]
actual_english_string <- paste(actual_english, collapse = " ")
english_lower <- tolower(actual_english_string)
english_words <- strsplit(english_lower, "\\W+")
sorted_freqs_english <- sort(table(english_words), decreasing = TRUE)
plot(sorted_freqs_english[1:10], type="b")
\end{verbatim}
\end{frame}

\begin{frame}[fragile]
\frametitle{First Solution, to make it work for all gutenberg.org texts}
\tiny
\begin{verbatim}
english <- scan("DollsHouse-Eng.txt", what = "character", sep = "\n")
meta.top.end <- which(startsWith(english,"*** START OF THIS PROJECT GUTENBERG EBOOK"))
meta.bottom.start <- which(startsWith(english, "End of the Project Gutenberg EBook"))
actual_english <- english[meta.top.end+1:meta.bottom.start+1]
actual_english_string <- paste(actual_english, collapse = " ")
english_lower <- tolower(actual_english_string)
english_words <- strsplit(english_lower, "\\W+")
sorted_freqs_english <- sort(table(english_words), decreasing = TRUE)
plot(sorted_freqs_english[1:10], type="b")
\end{verbatim}
\end{frame}

\begin{frame}[fragile]
\frametitle{Second Solution}
\tiny
\begin{verbatim}
library(stringr)
english <- scan("DollsHouse-Eng.txt", what = "character", sep = "\n")
meta.top.end <- which(str_detect(english,"START OF THIS PROJECT"))
meta.bottom.start <- which(str_detect(english,"END OF THIS PROJECT"))
actual_english <- english[meta.top.end+1:meta.bottom.start-1]
actual_english_string <- paste(actual_english, collapse = " ")
english_lower <- tolower(actual_english_string)
english_words <- strsplit(english_lower, "\\W+")
sorted_freqs_english <- sort(table(english_words), decreasing = TRUE)
plot(sorted_freqs_english[1:10], type="b")
\end{verbatim}
\end{frame}

\begin{frame}
\frametitle{Does this work?}
\begin{itemize}
\item Seems to work, but what if some book does not use upper case for those words? (Note: Gutenberg.org books are volunteer typing/proof reading efforts) \pause
\item We should perhaps look for some way of making the pattern case-insensitive
\item Or there is some other innovative solution. \pause
\item Working with real-world data is this process of trial and error. 
\item So, don't get frustrated. Persistence is the key. 
\end{itemize}
\end{frame}

\begin{frame}
\frametitle{General advice on working through this course}
\begin{itemize}
\item Follow the hands-on sessions - ask as many questions as you want, use the discussion forum
\item Get your hands dirty with lot of error messages in R studio
\item Carefully work through all the tutorial documents I share, or textbook lessons. 
\item It needs some extra time, and is not a regular humanities/social sciences class - accept that.
\item Believe that you will become text analysis novice-gurus soon!
\end{itemize}
\end{frame}

\begin{frame}
\frametitle{Last week's exercise and discussion}
\begin{itemize}
\item In Doll's House, how many times did: "you know" appear? \\
str\_count(dollshousetext,"you know") - gives 32
\item How many times did numbers  appear in the text (99712 is to be counted as 1 number. Not 5)?
\\ str\_count(dollshousetext, "\textbackslash\textbackslash d+") - gives 10
\end{itemize}
Note: dollshousetext here - is the output after paste() line in our original code i.e., the actual\_english\_string after removing all metadata.
\end{frame}

\begin{frame}
\frametitle{setwd()}
\begin{itemize}
\item How do we get the path? (from R-studio, from file browser on your computer i.e., my computer, finder etc)
\item Special caution for windows computers about paths: Forward vs Backward slash symbol
\item folder vs file
\item Why bother about setwd()? When is it not relevant? 
\end{itemize} \pause
Suggestions: \\
Go back and do Lesson 2 in Swirl(). \\ Another useful ref:  
\url{https://www.statmethods.net/interface/workspace.html}
\end{frame}
%how to set wd: navigate R file browser/look in your computer for path/else, just give the full path. 


\begin{frame}
\frametitle{}
\centering What is a corpus?
\end{frame}

\begin{frame}
\frametitle{Corpus}
\begin{itemize}
\item As a loose definition, we call the texts we want to analyze as the corpus.
\item A single piece of text is not a corpus. Corpus is a collection of such single texts. 
\item It can be 10 texts. It can be 100 texts. It can be 1000 texts. \pause
\item A collection of speeches converted into text (e.g., transcripts) is also a text corpus.
\item A collection of tweets is a corpus. \pause
\item As a working definition, we consider any collections of text we choose to work on to solve some problem as a corpus. 
\end{itemize}
\end{frame}

\begin{frame}
\frametitle{How do we get hold of a corpus?}
\begin{itemize}
\item You can a pre-existing corpus if that suits what you want to work on. 
\item Creating our own corpus to address our own problem:
\begin{itemize}
\item Data collection (downloading, scanning, web scraping etc)
\item Data preparation/pre-processing
\item Deciding on how to extract the information you need
\item Writing R code
\item Getting the results and analyzing them.
\end{itemize}
\end{itemize}
\end{frame}

\begin{frame}
\frametitle{Corpus pre-processing}
\begin{itemize}
\item Loading the corpus into R environment
\item Reading different formats of data in a way R can understand
\item Doing any other processing steps as needed (lower casing, sentence splitting etc)
\end{itemize}
\end{frame}

\begin{frame}
\frametitle{Loading corpus into R}
Before getting into this, we should know what are the different formats of text we can see online:
\begin{itemize}
\item plain text files - files that you can open and read in most of the text editors starting with basic ones such as notepad. 
\item PDF files - files you usually read in a pdf reader
\item Word documents and the likes -MS word etc.
\item XML documents 
\item HTML documents
\item tweets, tables etc (we will discuss as they come)
\end{itemize}
\end{frame}

\begin{frame}
\frametitle{PDF files}
\begin{itemize}
\item In the last class, we saw how to easily open a text file using scan() function in R and start doing some stuff like counting word frequencies right away. 
\item What happens if there is a PDF file instead of a text file in that scan() call. Did anyone try? \pause
\item Sometimes, you may end up with having only pdf versions of documents as your corpus (e.g., a corpus of research articles on topic X from XYZ journal)
\item What should we do in such cases? \pause
\item R supports pdf to text extraction. One of the libraries that can do this is: pdftools
\end{itemize}
\end{frame}

\begin{frame}[fragile]
\frametitle{Pdftools} 
\begin{itemize}
\item Install pdftools library by going to Tools $-->$ Install Packages and typing pdftools there.
\item Once installed, here is how you get a plain text version of a pdf file:
\begin{verbatim}
library(pdftools)
txt <- pdf_text("somefilename.pdf") 
\end{verbatim}
note: I am assuming this file is in my current working directory.  Else, we need to give full file path.
\pause \item This will not give you a cleanly formatted text string, you need to write extra code for that. This may also not work perhaps with all pdfs. Be aware of that.
\item If you end up using this for your work, make sure it works on your pdfs, and know what additional cleaning/formatting you should do within R to get some clean text.
\item More information on this library: \url{https://cran.r-project.org/web/packages/pdftools/pdftools.pdf} 
\end{itemize}
\end{frame}

\begin{frame}[fragile]
\frametitle{Doc files and qdaptools library}
\begin{itemize}
\item Install qdaptools library by going to Tools $-->$ Install Packages and typing qdaptools there.
\item Once installed, here is how you get a plain text version of a docx file:
\begin{verbatim}
library(qdapTools)
txt <- read_docx("somefile.docx")
\end{verbatim}
\end{itemize}
More functions: \url{https://cran.r-project.org/web/packages/qdapTools/qdapTools.pdf} - have a look at read\_docx section here. 
\end{frame}

\begin{frame}
\frametitle{XML files}
\begin{itemize}
\item XML stands for eXtensible Markup Language. 
\item It is a way to store and data with a tag structure.. so we can add a lot of metadata to text in this format. 
\item Although not extremely common, you may often end up having access to only XML versions of some corpus and asked to do something with it.
\item This can be done using XML library (note the upper case).
\end{itemize}
\end{frame}

\begin{frame}[fragile]
\frametitle{Reading XML files in R} \tiny
\begin{verbatim}
library(XML)
books <- xmlTreeParse("Books.xml", useInternalNodes = TRUE)
book_titles <- sapply(getNodeSet(books, "//catalog/book/title/text()"), xmlValue)
author_names <- sapply(getNodeSet(books, "//catalog/book/author/text()"), xmlValue)
description <- sapply(getNodeSet(books, "//catalog/book/description/text()"), xmlValue)
output <- data.frame(book_titles,author_names,description)
write.table(output, "xmlout.csv", sep="\t")
\end{verbatim}
\end{frame}

\begin{frame}
\frametitle{more on XML}
\begin{itemize}
\item I just took a simple example, but you can do many more complex things with XML package in R.
\item If you end up working with lot of xml files for some reason in future, remember to visit the package documentation: \url{https://cran.r-project.org/web/packages/XML/XML.pdf}
\item Another tutorial on XML parsing in R: \url{https://www.stat.berkeley.edu/~statcur/Workshop2/Presentations/XML.pdf}
\item I think Chapter 10 in textbook is complex to understand - so pointing to these resources.
\item Literature students - you may see more corpora in XML format, if we have to go by the author.
\end{itemize}
\end{frame}

\begin{frame}
\frametitle{HTML}
\begin{itemize}
\item HyperText Markup Language  - used to display webpages
\item the markup tells a webpage how to display text (font, bold, url, image etc)
\item Looks similar to XML, but different, and it serves a different purpose. 
\end{itemize}
\end{frame}

\begin{frame}[fragile]
\frametitle{Reading HTML files in R}
\tiny
\begin{verbatim}
library(XML)
url <- "http://radar.oreilly.com/2011/09/building-data-science-teams.html"
doc <- htmlParse(url, useInternal = TRUE)
links <-xpathSApply(doc, "//a/@href")
\end{verbatim}
\end{frame}

\begin{frame}
\frametitle{Assignment 1 - reminder/description}
\begin{itemize}
\item deadline: 27th Jan
\item Two questions - first one is writing a brief lit. review
\item Second question - small questions in R, which you should be able to do if you followed the class.
\end{itemize}
\end{frame}

\begin{frame}
\frametitle{Next Class}
\begin{itemize}
\item Continuation of today's topic
\item Some hands on practice
\item Storing your lines of code in .R files
\end{itemize}
\end{frame}

%Thursday
%Examples of reading a directory of texts using a loop.
%A2 is on reading HTML, reading through GuardianR 
%R Markdown usage. 

\end{document}
