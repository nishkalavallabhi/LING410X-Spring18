\documentclass{beamer}
\usepackage[utf8]{inputenc}
\usepackage{graphicx}
\author[Sowmya Vajjala]{Instructor: Sowmya Vajjala}

\title[LING 410X]{LING 410X: Language as Data}
\subtitle{Semester: Spring '18}

\date{25 January 2018}

\institute{Iowa State University, USA}
%%%%%%%%%%%%%%%%%%%%%%%%%%%

\begin{document}

\begin{frame}\titlepage
\end{frame}

\begin{frame}
\frametitle{Class outline}
\begin{itemize}
\item Some new R functions
\item Reading in multiple files from a folder
%\item Working with New York Times library in R
\item Storing programs in .R files
\item Group exercise
\item Reminder: Submit Assignment 1 on time.
\end{itemize}
\end{frame}

\begin{frame}
\frametitle{}
\Large Some new R functions
\end{frame}

\begin{frame}[fragile]
\frametitle{Adding more items to an existing vector}
\begin{verbatim}
days <- c("Monday", "Tuesday")
days <- c("Sunday", days)
days <- c(days,"Wednesday")
days
\end{verbatim}
\end{frame}

\begin{frame}[fragile]
\frametitle{setdiff function}
Returns difference between two vectors
\scriptsize
\begin{verbatim}
 v1 <- c("This", "is", "a", "small", "example")
 v2 <- c("is", "a")
 v3 <- setdiff(v1,v2)
\end{verbatim}
\small
v3 here will have "This", "small", "example"
\end{frame}

\begin{frame}[fragile]
\frametitle{Writing "loops" to repeat tasks}
\begin{itemize}
\item Why?: Tasks that need to be performed repetitively
\item E.g., I want to count 10 most frequent words not for one file, but for all files in a folder. \pause
\item R syntax for writing such loops - an example:
\begin{verbatim}
new <- vector()
for(day in days)
{
 new <- c(new,nchar(day))
}
new
\end{verbatim}
\end{itemize}
\end{frame}

\begin{frame}
\Large Reading a directory of text files - example
\end{frame}

\begin{frame}[fragile]
\frametitle{Iteratively accessing files in a folder-1}
\tiny
\begin{verbatim}
file.names <- dir("/home/bangaru/Desktop")
#this path is on my computer!!
for (f in file.names) {
print(f)
} 
\end{verbatim}
\end{frame}

\begin{frame}[fragile]
\frametitle{Iteratively accessing files in a folder-2}
\tiny
\begin{verbatim}
file.names <- dir(getwd(), pattern ="\\.txt")
for (f in file.names) {
print(f)
} 
\end{verbatim}
\end{frame}

\begin{frame}[fragile]
\frametitle{Reading the content of all files}
\tiny
\begin{verbatim}
setwd("my path to the required folder")
file.names <- dir(getwd(), pattern ="\\.txt")
file.names
data <- vector()
for (f in file.names) {
  tempData <- scan(f, what="character", sep="\n")
  tempDataString <- tolower(paste(tempData, collapse = " "))
  data <- c(data,tempDataString)
} 
#What will "data" contain now?
length(data)
data[1]
nchar(data[1])
nchar(data)
\end{verbatim}
\end{frame}


\begin{frame}[fragile]
\frametitle{continuing from there...}
Let us see if we can get the 10 most frequent words per book:
\tiny
\begin{verbatim}
for(item in data)
{
  words <- strsplit(item, "\\W+")
  sorted_freqs <- sort(table(words), decreasing = TRUE)
  print(sorted_freqs[1:10])
}
\end{verbatim}
\end{frame}

\begin{frame}[fragile]
\frametitle{continuing from there...}
Let us put everything in one loop: 
\tiny
\begin{verbatim}
setwd("my path to the required folder")
file.names <- dir(getwd(), pattern ="\\.txt")
for (f in file.names) {
  cat("in this file:",f, "\n")
  tempData <- scan(f, what="character", sep="\n")
  tempDataString <- tolower(paste(tempData, collapse = " "))
   words <- strsplit(tempDataString, "\\W+")
  sorted_freqs <- sort(table(words), decreasing = TRUE)
  print(sorted_freqs[1:10])
}
\end{verbatim}
\small -here, I did not do the part of reading all data into one larger variable - why? \\
(I uploaded a file called Directory.R which contains these lines of R code) \\
Note: There are other ways to do this in R. 
\end{frame}

\begin{frame}
\frametitle{storing .R files}
\begin{itemize}
\item Why?: Reusability - it wont go away when you close Rstudio.
\item Please note: On lab computers, it will be erased - so upload to box, google drive or some such storage before you leave.
\item Using a .R file:
\begin{itemize}
\item You can open in the left top panel.
\item Two options: source, run. Run runs the code on the console, line by line.
\item Source: file is stored and run internally and you just see the output. \pause
\item If you write your own R functions in your program, source("filename.R") will also allow other R functions you type in the console to use those from this file.
\end{itemize}
\end{itemize}
\end{frame}

\begin{frame}[fragile]
\frametitle{Exercise}
\begin{itemize}
\item Why are we lower casing, and what happens if we just don't remove punctuations, and just split by white space?
\item What sort of frequency tables will you see? 
\item How do these pre-processing decisions affect our results and analysis we do of them?
\item You can work in groups and discuss among yourselves, and post in the discussion forum any code (attach .R files) and comments on the outputs you see. 
\end{itemize}
\end{frame}


\begin{frame}
\frametitle{Next Week}
\begin{itemize}
\item Continuation and Conclusion of working with different formats (e.g., reading from NYTimes directly from R)
\item Recommended reading: \url{https://cran.r-project.org/web/packages/rtimes/rtimes.pdf}
\item Check the first 5-6 pages in this pdf, if possible, try to make it work in R Studio as shown in their example.
\item Reminder: Submit Assignment 1 \pause
\item Question: How many of you have Twitter accounts? How many are willing to create one for the class? 
\item If you are not interested, we won't do it as an exercise - I will just provide a tutorial file for enthusiasts.
\end{itemize}
\end{frame}

\end{document}



