\documentclass{beamer}
\usepackage[utf8]{inputenc}
\usepackage{graphicx}
\author[Sowmya Vajjala]{Instructor: Sowmya Vajjala}

\title[LING 410X]{LING 410X: Language as Data}
\subtitle{Semester: Spring '18}

\date{15 February 2018}

\institute{Iowa State University, USA}
%%%%%%%%%%%%%%%%%%%%%%%%%%%

\begin{document}

\begin{frame}\titlepage
\end{frame}

\begin{frame}
\frametitle{Class Outline}
\begin{itemize}
\item Announcements etc
\item Assignment 2 discussion
\item Discussion on the Technology Review article
\item Some notes on R syntax
\item Quick review of KWIC and n-grams
%\item Anatomy of a HTML page and how to work with it in R
%\item Discussion questions
\item Practice exercises.
\end{itemize}
\end{frame}

\begin{frame}
\frametitle{Extra tutorial session - optional}
\begin{itemize}
\item I want to hold an additional session on some evening (preferably thursday or friday) next week (say: 4-6pm) for general discussion and questions.
\item This is optional and there is no set agenda
\item The goal is to do some revision, and give additional clarifications on topics some of you are not clear about.
\item I will book a lab and let you know - you can walk in and join for as long as you want.
\item I would prefer if questions are sent to me apriori (I will setup a discussion forum)
\item How many of you will be interested in this? (It is okay if there are only 1 or 2)
\end{itemize}
\end{frame}

\begin{frame}
\Large Assignment 2 Discussion
\end{frame}

\begin{frame}[fragile]
\frametitle{Assignment 2 discussion: Question 1}
\tiny
\begin{verbatim}
htm <- readLines("2446.htm")
txt <- readLines("2446.txt")
length(htm) #gives 8139
length(txt) #gives 5081
\end{verbatim}
\normalsize
Questions:
\begin{itemize}
\item Why is the length different? 
\item Which one of these formats is easy to process forR in your opinion?  Why? 
\item Which one may give more interesting information?
\item readline() vs readLines() functions
\item readLi nes() vs scan() functions
\end{itemize}
\end{frame}

\begin{frame}[fragile]
\frametitle{Assignment 2 discussion: Question 2}
\tiny
\begin{verbatim}
guardian_key <- "XXXXXXX" 
step 1: results <- get_guardian("justin+trudeau", section= "world", api.key = guardian_key,
                           to.date = "2018-01-15", from.date = "2018-01-01")
#section is optional.
step 2: nrow(results) #gives number of results
step 3: names(results) #gives column names
step 3: 
my_df <- data.frame(results["id"],results["wordcount"])
#or whichever columns you want
step 4: copy-paste to spreadsheet, or use a new library to create spreadsheets from R or 
do this: write.csv(my_df, file = "results.csv")
- If you double-click this file you created, it usually will open in MS Excel 
or such software as a spreadsheet.
\end{verbatim}
\normalsize
\pause Oh, btw, I know many of your guardian API keys!!
\end{frame}


\begin{frame}
\Large Article Discussion
\end{frame}

\begin{frame}
\frametitle{Overview}
\begin{itemize}
\item What is already known  to the authors: ISIS uses social media, especially Twitter, to spread its ideas. \pause
\item Their question: What do ISIS and followers/sympathizers talk on twitter? Why (how?) do such messages spread? \pause
\item Data: about 2 million Arabic language tweets posted by 25K ISIS members over a period of: Jan-June 2015. \pause
\item Pre-processing: performed tokenization and stemming. Removed non-Arabic tweets. compiled 100 most popular stems. 
\item Further pre-processing: removed stems that are not related to their analysis. Left with 34 stems. Grouped them into four categories: violence, theological, sectarian, names.\pause
\end{itemize}
\end{frame}

\begin{frame}
\frametitle{Analysis}
\begin{itemize}
\item Analysis: if a tweet has majority stems from one category, categorize the tweet as that (main point. there is more).
\item They report on percentage of tweets belonging to different categories
\item They have plots showing the tweets of different categories over a period of time, and correlating them with news items. 
\end{itemize}
Note: I am not going to discuss their conclusions and implications and so on. That is not relevant for our course.
\end{frame}

\begin{frame}
\frametitle{}
\centering
\Large Quick notes on R syntax
\end{frame}

\begin{frame}[fragile]
\frametitle{collapse vs sep in paste function}
What will be output of all these???
\footnotesize
\begin{verbatim}
paste("1st", "2nd", "3rd", collapse = ", ")
paste("1st", "2nd", "3rd", sep = ", ")
paste("1st", "2nd", "3rd", collapse = ", ", sep = ":")
vec1 <- c("1st", "2nd", "3rd")
vec2 <- c("4th", "5th", "6th")
paste(vec1, collapse = ":: ")
paste(vec1, sep = ":: " )
paste(vec1, vec2, sep = "::")
paste(vec1, vec2, sep = "::", collapse = "--")
\end{verbatim}
\end{frame}

\begin{frame}[fragile]
\frametitle{Interesting R feature we need to be aware of}
\begin{verbatim}
paste(c('v1','v2'),collapes='+')
paste(c('v1','v2'),whatever='+')
\end{verbatim}
-What do you think happens in these two cases? \pause

paste interprets this as this: you want to paste each element of first vector with a variable collapes or whatever. Will not throw an error!!!
\end{frame}

\begin{frame}
\Large Tuesday class review
\end{frame}

\begin{frame}
\frametitle{Tuesday's class}
\begin{itemize}
\item Getting a word's occurrence, in context
\item KWIC.R, ModifiedKWIC.R - on canvas. You should figure out what the differences are. Both work.
\item We discussed about ngram package in R
\begin{itemize}
\item \tiny \url{https://cran.r-project.org/web/packages/ngram/ngram.pdf}
\item \tiny \url{https://cran.r-project.org/web/packages/ngram/vignettes/ngram-guide.pdf}
\end{itemize}
\end{itemize}
\end{frame}

\begin{frame}
\Large Practice Exercises
\end{frame}

\begin{frame}
\frametitle{Exercise - 1: KWIC}
\begin{itemize}
\item Modify the KWIC.R code from Tuesday to take the following information from a user: a folder/directory path, a word, context size. 
\item Using this information, print to the R console the word and its context for all .txt files in the folder, one by one.
\item Note 1: create a small folder with 2 or 3 files. Don't try with 100 files directly.
\item Note 2: Don't test using a common word like "The" or something. You will see a lot of output! Try with some very infrequent word.
\item Share your code on the forum for today.
\end{itemize}
\end{frame}

\begin{frame}
\frametitle{Exercise - 2: Ngrams}
Take any text file, get top-10 uni/bi/tri/4 ngrams for this file. Create four 4 plots where x-axis is 1--10, y-axis has the ngram frequency (i.e., plot the frequencies of ngrams in descending order). Do this with R markdown and post your html/doc/pdf on discussion forum titled 15th Feb 2018. 
\end{frame}

\begin{frame}
\frametitle{Additional Exercise, if interested}
Modify the KWIC program to suit variable contexts on left and right (e.g., 2 words left, 3 words on right)
\end{frame}

\begin{frame}
\frametitle{Next week}
\begin{itemize}
\item Topics: overview of text analysis and pattern extraction from collections of documents - text classification, clustering and topic modeling
\item Introduction to text classification
\item To do for you: read this article: \url{https://goo.gl/qhT3u4} and we will discuss about this next week on Tuesday during the class. 
\item Keep this in mind: We are moving from straight forward counting analyses to predictive analysis from next week!
\end{itemize}
\end{frame}

\end{document}
