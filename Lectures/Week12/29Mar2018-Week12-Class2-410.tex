\documentclass{beamer}
\usepackage[utf8]{inputenc}
\usepackage{graphicx}
\author[Sowmya Vajjala]{Instructor: Sowmya Vajjala}

\title[LING 410X]{LING 410X: Language as Data}
\subtitle{Semester: Spring '18}

\date{29 March 2018}

\institute{Iowa State University, USA}
%%%%%%%%%%%%%%%%%%%%%%%%%%%

\begin{document}

\begin{frame}\titlepage
\end{frame}

\begin{frame}
\frametitle{Topic modeling -resources page}
\url{https://github.com/trinker/topicmodels_learning}
\end{frame}

\begin{frame}
\frametitle{"Learning" a topic model with mallet: steps in the code}
\begin{enumerate}
\item Pre-processing of the corpus (in author's version: splitting it into chunks, pre-processing of chunks)
\item Getting it into a two column format (id, text).
\item Using mallet package in R to build a topic model.
\item use mallet.import() function to convert a dataset into mallet format
\item use MalletLDA() function to create a topic model, setting its parameters. 
\item observe and analyze the output in different ways. 
\end{enumerate}
\end{frame}

\begin{frame}
\frametitle{Tuesday's class}
\begin{itemize}
\item I followed this step by step, with movie reviews corpus (without considering positive/negative sentiment information)
\item Code for that is: anothertopicmodelexample.R on Canvas.
\item We also discussed about how a topic model learns (conceptually), and how are they typically evaluated.
\item Today: it is more about building your own topic models.
\end{itemize}
\end{frame}

\begin{frame}
\frametitle{Exercise}
\begin{itemize}
\item open mallet-29marlab.R - you have some lines of code, along with comments and questions.
\item Walk through that step by step (dataset is provided) and answer the questions I posed there.
\item About the corpus: common core standards exemplar texts - from US school textbooks/reading materials. 
\item Different topics, different genres - literature, informational texts, speeches etc. 
\item You may recognize some titles.
\end{itemize}

Post your observations from the exercise, addressing my questions in the comments inside R code. Post in the forum with today's date. 
\end{frame}

\begin{frame}
\frametitle{Exercise: Alternate option}
\begin{itemize}
\item Topic modeling with News articles.
\item Get news articles on "USA" from Guardian for two time periods. Build one topic model per time period based on article content.
\item Compare the topics discussed about USA between different time periods.
\item Same questions as before remain about pre-processing, choosing number of topics etc. 
\end{itemize}
\end{frame}

\begin{frame}
\frametitle{General Remarks for next week}
\begin{itemize}
\item We completed all except two chapters from the textbook (chapter 5, chapter 11).
\item Next week: text visualization (covers ideas in Chapter 11) - read this chapter. 
\item Reminder: Submit Assignment 5 
\item Regarding Assignment 6: You can either use mallet library or use topicmodels library as described in the tutorial. 
\item Start thinking about your final projects now - conceptually, everything is pretty much done.
\end{itemize}
\end{frame}

\end{document}
