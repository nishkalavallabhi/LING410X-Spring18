\documentclass{beamer}
\usepackage[utf8]{inputenc}
\usepackage{graphicx}
\author[Sowmya Vajjala]{Instructor: Sowmya Vajjala}

\title[LING 410X]{LING 410X: Language as Data}
\subtitle{Semester: Spring '18}

\date{12 Apr 2018}

\institute{Iowa State University, USA}
%%%%%%%%%%%%%%%%%%%%%%%%%%%

\begin{document}

\begin{frame}\titlepage
\end{frame}

\begin{frame}
\frametitle{Class Outline}
\begin{itemize}
\item Plan, Schedule for final project presentations (24th, 26th April)
\item General practice 
\item Reminder: Submit Assignment 6 on time! 
\end{itemize}
\end{frame}

\begin{frame}
\frametitle{Final project presentations}
\begin{itemize}
\item 5\% of your total grade
\item 10-12 min maximum + 3-5min for questions (per team, if you are doing together with someone)
\item What I expect: 
\begin{enumerate}
\item Describe what you are trying to do, how is it relevant to outside world, to your own discipline etc.
\item Show us what you have so far (some figures, if you have any, is good)
\item Talk about how you got those results you show
\end{enumerate}
\end{itemize}
\end{frame}

\begin{frame}
\frametitle{Presentations: schedule for 24th April}
In this order:
\begin{enumerate}
\item Carlos Eduardo Back Vianna
\item Su-Yeon Cho
\item Brody Dingel and John Piegors
\item Helena Hansen
\item Gage Williams
\end{enumerate}
\end{frame}

\begin{frame}
\frametitle{Presentations: schedule for 26th April}
In this order:
\begin{enumerate}
\item Lauren Didesch
\item Kori Ralston
\item Xiaochi Jin and Nicole Richwine
\end{enumerate}
\end{frame}

\begin{frame}
\frametitle{}
Three exercises are described in the next few slides. You can choose whichever is closer to your final project idea and work on that. You can also work on your final projects in the class if you want. But work on only stuff related to this class! \bigskip

Write a summary of what you did today in the discussion forum for today. 
\end{frame}


\begin{frame}
\frametitle{Exercise on Text Clustering/Visualization}
\begin{itemize}
\item There are two zip files-stylocorpus.zip and rollingdelta.zip - learn to work with Stylo library in R, and using these datasets. Follow the stylo tutorials from last time.
\item Note down your observations. 
\end{itemize}
\end{frame}

\begin{frame}
\frametitle{Exercise on Text Classification}
\begin{itemize}
\item There is a zip file prof-classify.zip containing the data. primary\_set folder contains about 700 text files, and there are two categories: A2 and B2. Ignore all other information in the file name. Consider this as your training data. There are some test instances in secondary\_set.
\item Source of data: \url{https://goo.gl/v0r6et}
\\ (we are using a subset of the full data today)
\item Description of data: It is a dataset of essays written by English learners from China, Japan and Korea, and there are two levels of proficiency (A2 and B2)
\item Task: learn to distinguish between A2 and B2 - use any classification library you want (including stylo)
\item Data in stylo format is in: forStylo folder. General data, which you can use with tm or anything is in the other folder. 
\item Note down your observations. 
\end{itemize}
\end{frame}

\begin{frame}
\frametitle{Exercise on Topic Modeling}
\begin{itemize}
\item Use any collection of files you want, practice doing topic models 
\item Note down your observations. 
\end{itemize}
\end{frame}

\begin{frame}
\frametitle{Next week}
\begin{itemize}
\item Attendance for today: Write about what you did today.
\item Topics for next week: Revision. 
\item Post on the discussion board in a thread called "Topics for Revision" if you want any topics to be discussed in better detail in the coming week.
\end{itemize}
\end{frame}

\end{document}

