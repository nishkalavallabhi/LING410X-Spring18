\documentclass{beamer}
\usepackage[utf8]{inputenc}
\usepackage{graphicx}
\author[Sowmya Vajjala]{Instructor: Sowmya Vajjala}

\title[LING 410X]{LING 410X: Language as Data}
\subtitle{Semester: Spring '18}

\date{6 February 2018}

\institute{Iowa State University, USA}
%%%%%%%%%%%%%%%%%%%%%%%%%%%

\begin{document}

\begin{frame}\titlepage
\end{frame}

%MENTION THIS: 

\begin{frame}
\frametitle{Class Outline}
\begin{itemize}
\item Recap of last week
\item A note on the various data structures in R
\item Calculating lexical richness in a text (Chapters 6 in Textbook)
\item Reminder: Assignment 2 submission is due on 10th Feb.
\item news: \url{https://www.r-bloggers.com/analysis-of-trumps-state-of-the-union-speech-with-r/}
\end{itemize}
\end{frame}

\begin{frame}
\frametitle{Recap of last week-1}
\begin{itemize}
\item Text analysis questions: Analyzing word usage in different parts of a text (chapter by chapter, word by word etc)
\item R specific concepts:
\begin{enumerate}
\item Differences between a vector and a list
\item Usage of lapply 
\item Writing a for- loop
\item Replacing NAs with zero
\end{enumerate}
\end{itemize}
\end{frame}

\begin{frame}[fragile]
\frametitle{Recap of last week-2}
My Question: Using R program, try to find out:
\begin{itemize}
\item How many chapters are there in David Copperfield (\url{http://www.gutenberg.org/files/766/766-0.txt})? 
\item What chapter has the most number of occurrences of the name Murdstone? \pause
\item My solution for 1st question (I actually do not need all that other stuff to answer just this question!)
\scriptsize
\begin{verbatim}
davidc <- scan("davidc.txt", what = "character", sep = "\n")
davidc.chap.positions <- grep("^CHAPTER \\d", davidc)
length(davidc.chap.positions)
\end{verbatim} \small
\item For second part, Brody will explain how he found out. 
\item Additionally, you can check: 01FebSolution.R for today's canvas (.. and modify it according to your specifications)
\end{itemize}
\end{frame}

\begin{frame}
\Large Data Structures in R
\end{frame}

\begin{frame}
\frametitle{What is a data structure?}
\begin{itemize}
\item a datastructure refers any way of organizing data.
\item examples: a list, a vector, a matrix, a table etc.
\item purpose of a data structure is to store data in some organized manner such that it is easy to access it later 
\item What data structure you need will depend on how your input and output should look like.
\item When we use pre-defined R functions such as sort, lapply etc., we should know what do they take as input, and what they give as output. 
\end{itemize}
\end{frame}

\begin{frame}
\frametitle{R Datastructures: Lists and Vectors}
%start with a question.
\begin{itemize}
\item What is a vector?
\item What is a list?
\item What is the difference between vector and list? \pause
\item What does a sort() function take as input? a vector or a list? How can I find out? \pause
\item What does seq(1:5) return? a list or a vector? How can I find out? 
\end{itemize}
\end{frame}

\begin{frame}[fragile]
\frametitle{R Datastructures: Matrices and Arrays}
\begin{itemize}
\item A matrix is a 2 dimensional datastructure (arranged in rows and columns) where all elements are of the same data type (integers or strings or decimal numbers etc)
\item If we extend this idea to more than 3-dimensions, we have a data structure called array.
\item How does a matrix look like? Try this:  matrix(1:6, ncol=3)
\item How does an array look like? Try this for a 3-D array: array(1:8, dim=c(2, 2, 2))
\end{itemize}
\end{frame}

\begin{frame}[fragile]
\frametitle{R Datastructures: Data Frames}
\begin{itemize}
\item A "Data Frame" is useful to combine several vectors together as columns in a table. 
\item We use data.frame() function to create a data frame. \pause
\item Let us look at this example:
\footnotesize
\begin{verbatim}
line_number <- seq(1:5)
names <- c("Santa", "Banta", "Manta", "Tanta", "Anton")
college <- c("LAS", "Business", "Engineering", "LAS", "VetMed")
df <- data.frame(line_number,names, college)
\end{verbatim}
\end{itemize}
\end{frame}


\begin{frame}
\frametitle{R Datastructures: Factors}
\begin{itemize}
\item A "Factor" is a data structure useful to store categorical data (what is that?) \pause
\item In my previous data frame example, the column college has 5 values, but 4 unique values. 
\item The 4 unique values can be seen as different factors/categories of colleges. 
\item We use factor() function to work with factor data structure.
\item Example usage: factor(df\$college) from previous slide.
\end{itemize}
\end{frame}

\begin{frame}
\Large Lexical Variety
\end{frame}

\begin{frame}
\frametitle{What is lexical variety or vocabulary richness}
\begin{itemize}
\item Lexical variety refers to some quantification of the diverseness of the vocabulary used in a text/corpus.
\item There are two common ways of defining lexical variety:
\begin{itemize}
\item Mean word frequency: average frequency of words in a document.
\item Type-token ratio: (number of unique words/total number of words)*100
\end{itemize} \pause
\item Let us say this is the text I have: "Here is a sentence with sentence coming twice".
\item What is the mean-word frequency? What is the type-token ratio? \pause
\end{itemize}
\end{frame}

\begin{frame}[fragile]
\frametitle{Calculating Lexical Variety}
Let us start from last week's example again. 
\tiny
\begin{verbatim}
moby <- scan("mobydick.txt", what = "character", sep = "\n")
moby.start <- which (moby ==  "CHAPTER 1. Loomings.")
moby.end <- which (moby == "orphan.")
moby.actual <- moby[moby.start:moby.end]
moby.chap.positions <- grep("^CHAPTER \\d", moby.actual)
moby.actual[moby.chap.positions]

moby.last.position <- length(moby.actual)
moby.chap.positions <- c(moby.chap.positions, moby.last.position)
moby.actual[moby.chap.positions]

chapters.raw <- list()
for (i in 1:(length(moby.chap.positions) -1))
{
  titleline <- moby.chap.positions[i]
  title <- moby.actual[titleline]
  start <- titleline+1
  end   <- moby.chap.positions[i+1]-1
  chapter.lines <- moby.actual[start:end]
  chapter.string <- tolower(paste(chapter.lines, collapse = " "))
  chapter.string <- gsub(" +", " ", gsub("[[:punct:]]", " ", chapter.string))
  chapter.words <- unlist(strsplit(chapter.string, "\\W"))
  chapter.freqs <- table(chapter.words)
  chapters.raw[[title]] <- chapter.freqs
}
\end{verbatim}
\end{frame}

\begin{frame}[fragile]
\frametitle{Mean Word Frequency}
\begin{itemize}
\item Let us say I want to calculate mean (average) word frequency for each chapter in Mobydick. 
\item What information do I need for this?  \pause
\item Okay, let us take one chapter as an example.
\tiny
\begin{verbatim}
sum_of_all_word_freqs_chapter1 <- sum(chapters.raw[[1]])
num_words_in_chapter1 <- length(chapters.raw[[1]])
mean_word_freq_chapter1 <- sum_of_all_word_freqs_chapter1/num_words_in_chapter1
another_way_for_this <- mean(chapters.raw[[1]])
ttr_chapter1 <- (num_words_in_chapter1/sum_of_all_word_freqs_chapter1)*100
\end{verbatim}
\small
\item If we do this for all chapters, we get the mean word frequency per chapter. 
\end{itemize}
\end{frame}

\begin{frame}[fragile]
\frametitle{How do we get mean word frequency for all chapters?}
\begin{itemize}
\item Use a for loop like last week:
\tiny
\begin{verbatim}
means = c()
for(i in 1:length(chapters.raw))
{
 means[i] <- sum(chapters.raw[[i]])/length(chapters.raw[[i]]) 
}
\end{verbatim} \pause
\normalsize
\item Use lapply: lapply(chapters.raw, mean)
\item What does this give? 
\end{itemize}
\end{frame}

\begin{frame}[fragile]
\frametitle{unname and unlist}
\begin{itemize}
\item lapply returns a list. unlist() converts it into a vector (plotting function in the next slide looks for a vector).
\item unname() removes the chapter names and retains only the numbers (Not needed here, Just doing to show this)
\tiny
\begin{verbatim}
means <- unlist(means)
means <- unname(means)
\end{verbatim} \small
\item At this point, means is just a vector of numbers, which we can use to plot.
\pause \item Instead of using lapply + unlist, we can use sapply() which directly returns a vector. 
\end{itemize}
\normalsize note: textbook uses rbind and do.call() functions to do the same. 
\end{frame}

\begin{frame}[fragile]
\frametitle{Plotting the mean word frequency}
\tiny
\begin{verbatim}
par(mfrow=c(1,2)) 
plot(means, type="h")
plot(scale(means), type = "h")
\end{verbatim} \normalsize
Look at the two plots - what are the differences? What is scale() doing? 
\end{frame}

\begin{frame}
\frametitle{scaling}
\begin{itemize}
\item Scaling a numeric vector means subtracting the average value from each number in the vector
\item So, if we plot the scaled vector, it will show us the deviations from average. \pause
\item In this case, the positive values after scaling indicate the chapters which have a higher mean word frequency than the overall mean.
\item negative values indicate chapters which have a lower mean word frequency than the overall mean. \pause
\item if mean word frequency is used as a measure of lexical difficulty, we can say those positive valued chapters are difficult.  
\end{itemize}
\end{frame}


\begin{frame}
\frametitle{Thursday}
\begin{itemize}
\item R: Writing our own functions in R
\item Creating R markdown reports (like my tutorial pdfs)
\item Practice what we learnt so far
\end{itemize}
%And an activity with creating R markdown of some basic analysis: can give an assignment about twitter.
\end{frame}

\end{document}

%What is a data structure
%https://www.stat.auckland.ac.nz/~paul/ItDT/HTML/node64.html

%Day 1: Chapter 6,7
%Day 2: Chapter 8--9, R Markdown, Assignment 3 description

%10min: Continue on twitter
%20 min: data structures
%30min: lex. variety, hapax richness
%15min: A3 description
