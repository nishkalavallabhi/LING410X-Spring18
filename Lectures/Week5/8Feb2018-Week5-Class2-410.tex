\documentclass{beamer}
\usepackage[utf8]{inputenc}
\usepackage{graphicx}
\author[Sowmya Vajjala]{Instructor: Sowmya Vajjala}

\title[LING 410X]{LING 410X: Language as Data}
\subtitle{Semester: Spring '18}

\date{8 February 2018}

\institute{Iowa State University, USA}
%%%%%%%%%%%%%%%%%%%%%%%%%%%

\begin{document}

\begin{frame}\titlepage
\end{frame}


\begin{frame}
\frametitle{Class Outline}
\begin{itemize}
\item writing R functions
\item Rmarkdown: introduction and practice exercise
\item Reminder: Assignment 2 due on 10th!
\end{itemize}
\end{frame}

\begin{frame}
\frametitle{Interesting stuff I found}
\begin{itemize}
\item a 1994 article that did text analysis on the "Book of Genesis"! \\ 
\url{http://projecteuclid.org/download/pdf_1/euclid.ss/1177010393}
\end{itemize}
\end{frame}

\begin{frame}
\Large Writing R functions
\end{frame}

\begin{frame}
\frametitle{What is a function? Why use it?}
\begin{itemize}
\item Functions are reusable pieces of code you can just "call" and use instead of writing everything line by line.
\item sort(), unlist(), table() and all these things you saw are such "built-in" functions. \pause
\item Let us say you want to sort a list of words and frequencies in decreasing order. 
\item You don't now start writing code for the sorting process from scratch. 
\item You just use sort() function, and you can use it again and again. 
\item If we already have so many functions in R, why write new ones? \pause
\item To do some custom tasks that we want, for which some such function does not already exist in R.
\end{itemize}
\end{frame}

\begin{frame}
\frametitle{Functions vs Loops - your responses}
\begin{itemize}
\item In some sense, a loop is also some sort of function to repeatedly do something.
\item A loop can call some functions repeatedly inside it
\item A function can have loops inside it \pause
\item Difference is that: your function, once defined and saved somewhere on your computer, can be "imported" into any R code you are writing at a later point in time, and used as if it is a built-in R function. \pause
\item With loops, you cannot do that unless you put that loop inside a function. 
\end{itemize}
\end{frame}

\begin{frame}[fragile]
\frametitle{Writing a function - Example}
\scriptsize
\begin{verbatim}
my_square_function <- function(number)
{
  return(number * number)
}

my_square_function(4) #Gives 16
\end{verbatim}
\end{frame}

\begin{frame}[fragile]
\frametitle{Writing a function - Example}
\scriptsize
\begin{verbatim}
my_number_function <- function(number)
{
  return (c(number*number, number*number*number, number*number*number*number))
}

my_number_function(4) #Gives a vector with values 16, 64, 256
\end{verbatim}
\end{frame}

\begin{frame}[fragile]
\frametitle{So, how do I use functions?}
\begin{itemize}
\item Let us take this example from last week, for creating dispersion plots.
\item When I wanted to create a dispersion plot for the word "nora" in Dolls House, here is what I did:
\tiny
\begin{verbatim}
progress <- seq(1:length(dollshouse_words_vector))
nora <- which(dollshouse_words_vector == "nora")
length(nora)
nora_progression <- rep(NA, length(progress))
nora_progression[nora] <- 1
plot(nora_progression, main="Dispersion plot for word 'nora' in 'A Doll\'s House' play",
    xlab="position in text", ylab="nora", type="h", ylim=c(0,1), yaxt = 'n')	
\end{verbatim}
\end{itemize}
\end{frame}

\begin{frame}[fragile]
\frametitle{Where are functions useful?}
When I wanted to repeat the same process for another word "rank", here is what I had to do:
\tiny
\begin{verbatim}
progress <- seq(1:length(dollshouse_words_vector))
rank <- which(dollshouse_words_vector == "rank")
rank_progression <- rep(NA, length(progress))
rank_progression[rank] <- 1
plot(rank_progression, main="Dispersion plot for word 'rank' in 'A Doll\'s House' play",
     xlab="position in text", ylab="rank", type="h", ylim=c(0,1), yaxt = 'n')
\end{verbatim}
\normalsize
If I had a third word, I should do all these again. What if I can put this piece of code as a custom function, and use that function wherever I want?
\end{frame}

\begin{frame}[fragile]
\frametitle{Writing a function for dispersion plot}
\tiny
\begin{verbatim}
get_dispersion_plot <- function(wordsvector, word)
{
  progress <- seq(1:length(wordsvector))
  word_presence <- which(wordsvector == word)
  length(word_presence)
  word_progression <- rep(NA, length(progress))
  word_progression[word_presence] <- 1
  plot(word_progression, main="Dispersion plot for the given word in 'A Doll\'s House' play",
    xlab="position in text", ylab=word, type="h", ylim=c(0,1), yaxt = 'n')
}
\end{verbatim}
\normalsize Now, I can use this dispersion function to get the plot for any word, once I create that vector of words. 
\\- Note that this is not a loop. 
\end{frame}

\begin{frame}[fragile]
\frametitle{A function for getting the word vector from a file}
\tiny
\begin{verbatim}
get_words_vector <- function(file_path)
{
  fulltext <- scan(file_path, what = "character", sep = "\n")
  fulltext_as_string <- paste(fulltext, collapse = " ")
  words_vector <- unlist(strsplit(tolower(fulltext_as_string), "\\W+"))
  return (words_vector)
}
\end{verbatim}
\end{frame}

\begin{frame}[fragile]
\frametitle{Using these two functions}
\begin{itemize}
\item Let us say I put these functions in a file called TextAnalFunctions.R.
\item Using these functions in R console follows the following steps:
\begin{enumerate}
\item Set the working directory to where your file is (else, use the full path)
\item First, I "source" or load the R file with my functions
\item use get\_words\_vector to get words vector for the file
\item use that words vector from the previous step to get dispersion plot for a given word from this vector.
\end{enumerate}
\item in R \begin{verbatim}
source("TextAnalFunctions.R").
words <- get_words_vector("DollsHouse-Eng.txt")
get_dispersion_plot(words, "nora") 
#or whatever word you want. 
\end{verbatim}
\end{itemize}
\end{frame}

\begin{frame}
\frametitle{Functions: Conclusion}
\begin{itemize}
\item Functions also make your R code more readable.
\item When to use (and whether to use) functions - depends on what you are doing.
\item More examples: \url{http://www.statmethods.net/management/userfunctions.html}
\end{itemize}
\end{frame}

\begin{frame}
\Large Using functions to find Hapax Legomena (Chapter 7)
\end{frame}

\begin{frame}
\frametitle{Hapax Legomena}
\begin{itemize}
\item We looked at average word frequency and TTR on Tuesday. 
\item Another way of looking at vocabulary richness is to look at the number of words that occur very infrequently in the text. 
\item If we consider words that appeared only once, we call them singleton/one-zies/hapax legomena
\item How do you get such information? There is no such pre-defined function like mean() or sum() to return frequencies that are 1. 
\end{itemize}
\end{frame}

\begin{frame}[fragile]
\frametitle{sapply, with custom function definition}
Consider this line below (chapters.raw - is the variable from our last class): \tiny
\begin{verbatim}
hapax <- sapply(chapters.raw, function(x) sum(x == 1)) 
\end{verbatim} \normalsize
Alternatively, the following also works: \tiny
\begin{verbatim}
hapaxfunction <- function(x)
{
  return(sum(x ==1))
}
hapax <- sapply(chapters.raw, hapaxfunction) 
\end{verbatim}
-What this says is: for each item in chapters.raw, i.e., for each chapter, count the number of words whose frequency is 1. 
\end{frame}

\begin{frame}[fragile]
\frametitle{From what we saw so far...}
What will these lines below do?: \tiny
\begin{verbatim}
hapax <- sapply(chapters.raw, function(x) sum(x == 1))
hapax <- unname(hapax)
lengths <- unname(sapply(chapters.raw, sum)) #What will this have?
new <- hapax/lengths
\end{verbatim} \normalsize
-What will "new" have eventually??
\end{frame}

\begin{frame}
\Large R Markdown
\end{frame}

\begin{frame}
\frametitle{What is R Markdown?}
\begin{itemize}
\item R markdown is a formatting system to create HTML, PDF, or Word reports that use R code.
\item Why use it?: Typically used to share your R based analysis and results with others. Supports reproducible research.
\item Advantage: R code can be embedded inside the report. So, you can just keep adding R code, its output, and your comments, and prepare a neatly formatted report. \pause
\item This is how all those tutorial documents I shared have been created. 
\end{itemize}
\end{frame}

\begin{frame}
\frametitle{How to start?}
\begin{itemize}
\item First, install the Rmarkdown package by going to tools$->$install packages and selecting rmarkdown or by typing: install.packages("rmarkdown") on R console. \pause
\item Now, in your R studio, go to File $->$ New $->$ Rmarkdown. Give some title to your report, write your name, and create a document. \pause
\item There are three major types of data in this - it starts with some metadata about the document, and then there is either Rcode or some plain text.
\item Resource: \url{http://rmarkdown.rstudio.com/lesson-1.html}
\end{itemize}
\end{frame}

\begin{frame}
\frametitle{R markdown exercise}
\begin{itemize}
\item With this elementary introduction to Rmarkdown, do a small exercise (individually or as groups)
\item Prepare a R markdown report that shows the top ten words in any text file you take as input (sample text file provided) 
\item It need not necessarily be from Gutenberg.org - so, you don't have to handle the removal of metadata part. 
\item Note: We already did this before, and you also have my Tutorial document as a reference to do this.
\item Once you are done, go to Canvas course page, open Discussion forums, look for a forum titled 8thFeb2018 and post your doc (or html or pdf) reports there along with the .Rmd file you created.
\end{itemize}
\end{frame}

\begin{frame}
\frametitle{Next Week}
\begin{itemize}
\item Text analysis topics: 
\begin{itemize}
\item Searching for keywords and their contexts inside texts (Read chapters: 8--9 in textbook)
\item Moving beyond single words and looking at word pairs, word triplets etc.
\end{itemize}
\item R specifics: creating working with .R files. Continuing to work with R functions.  
\item For those who want to do more: 
\begin{enumerate}
\item Convert all text analyses we did so far into functions and store them in one file, calling it text analysis or something.
\item Use loops to loop through all files in a folder, and repeat whatever text analyses you want by calling these functions in the loop.
\end{enumerate}
\end{itemize}
\end{frame}

\end{document}

\begin{frame}
\frametitle{Optional second exercise}
\framesubtitle{... for those who finished the first one}
\begin{itemize}
\item Get the 1000 most recent tweets on "Microsoft" from Twitter and extract the following information from them:
\begin{itemize}
\item number of unique users in these tweets
\item percentage of retweets in these tweets
\item average number of words in each tweet
\end{itemize}
\item Try to create a report using R markdown for this as well (without making your twitter access tokens and stuff visible to readers!)
\end{itemize}
\end{frame}

