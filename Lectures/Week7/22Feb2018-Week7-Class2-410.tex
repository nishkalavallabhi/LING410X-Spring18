\documentclass{beamer}
\usepackage[utf8]{inputenc}
\usepackage{graphicx}
\author[Sowmya Vajjala]{Instructor: Sowmya Vajjala}

\title[LING 410X]{LING 410X: Language as Data}
\subtitle{Semester: Spring '18}

\date{22 February 2018}

\institute{Iowa State University, USA}
%%%%%%%%%%%%%%%%%%%%%%%%%%%

\begin{document}

\begin{frame}\titlepage
\end{frame}

\begin{frame}
\frametitle{Outline}
\begin{itemize}
\item Review of last class
\item Text classification - quick summary of the process (textbook)
\item Individual or Group activity - summarizing what we did so far
\item Reminder: Tutorial at 5pm, in this room. 
\end{itemize}
\end{frame}

%ADD: Ask to do swirl exercises - revisit!!

\begin{frame}
\Large Review of last class
\end{frame}

\begin{frame}
\frametitle{Text Classification}
\begin{itemize}
\item In text classification, we know what the possible categories for our texts can be.
\item We also have a collection of texts, already assigned these categories. \pause
\item We want to create a "model" of categorization based on this pre-categorized text collection such that the model can "predict" or "assign" categories to new texts.
\end{itemize}
\end{frame}

\begin{frame}
\frametitle{classification and clustering}
\begin{itemize}
\item Similarity: we have a collection of texts, we know what are the features we want to look for (words) \pause 
\item Difference: In the case of classification, we have a few examples classified into some categories, and we want the machine to learn to classify like this for new texts. \pause
\item In clustering, we do not have such examples, we have no idea how many groupings or possible. We want the machine to figure that out AND do the grouping. 
\end{itemize}
\end{frame}

\begin{frame}
\frametitle{Topic Modeling}
\begin{itemize}
\item Idea 1: each document is a mixture of topics
\item Idea 2: each topic can be represented by groups of words associated with it. \pause
\item Idea 3: a word may be very important in one topic, but may not be so important in another topic \pause
\item So how about looking at a collection of documents, extracting main topics from them and forming clusters of words based on topical similarity?
\end{itemize}
\end{frame}


\begin{frame}
\frametitle{}
\Large Assigned Readings from last class
\end{frame}

\begin{frame}
\frametitle{Reading 1}
\begin{itemize}
\item They discuss the question of automatically identifying the style of writing in newspapers, using text classification
\item Data: Four collections (categories) of newspaper articles
\item Features: Function words (idea - function word usage is not driven by topic, and hence, truly captures style); POS tag frequencies. 
\item Learning method: a standard classification algorithm in those days (Ripper)
\end{itemize}
\end{frame}

\begin{frame}
\frametitle{Reading 1 - What did they find out?}
\begin{itemize}
\item There are very few features that distinctively identify one category or the other.
\item So, what if a given document just does not have those features? 
\item conclusion: they say they can identify signature features that work for some cases at some times, but not all cases at all time. \pause
\item Note: This is very early work on text classification. It is still a very active area of interest to researchers across disciplines and practitioners in technology industry. 
\end{itemize}
\end{frame}

\begin{frame}
\frametitle{Reading 2}
\begin{itemize}
\item Machine learning: the method of showing computers a lot of examples of something, so that it can "learn" from those examples. \pause
\item Classification is a form of machine learning. \pause
\item Machine learning is a part of many technologies you use today (email, search, your mobile phone apps, siri/cortana etc) \pause
\item Applied Linguistics primarily focuses on topics related to teaching, learning languages, instructional settings, assessment by tests etc. 
\item Machine Learning (specifically text classification) is useful for:
\begin{enumerate}
\item reading/writing/listening/speaking assessment (e.g., in GRE/TOEFL etc)
\item developing tools that are useful for language learners (e.g., grammarly.com)
\item doing several other tasks such as educational data mining
\end{enumerate}
\end{itemize}
\end{frame}

\begin{frame}
\frametitle{Something I came across yesterday}
\begin{itemize}
\item Title: "Women better represented in Victorian novels than modern, finds study"
\item Source: The Guardian (\url{https://goo.gl/uwkis4})
\item Quick summary - Data: "An analysis of more than 100,000 novels spanning more than 200 years"
\item Methods: They used a software called BookNLP, which processes books (.txt files too!), identifies references to characters and groups them, does gender identification (i.e., a form of classification) etc.
\item Hypothesis: "expected to see an increase in the prominence of female characters in literature across the two centuries"
\item What Data told: "from the 19th century through the early 1960s we see a story of steady decline"
\end{itemize}
\end{frame}

\begin{frame}
\frametitle{}
\Large Doing Text Classification in R
\\ \tiny (quick summary of chapter 12 in the textbook)
\end{frame}

\begin{frame}
\frametitle{What is the data?}
\begin{itemize}
\item There is a collection of books written by about 12 authors. 42 books in total.
\item There is one anonymous text. 
\item Task is to "learn" to classify between authors, and use the classification model they learnt to predict the authorship of the anonymous text.
\pause \item But there are only 42 texts for all authors together, which is not really suitable for teaching a machine to learn to classify. 
\item Text classification typically needs 100s of examples per category. 
\item How did he handle this issue?: he decided to split each file into 10 equal parts, and treat each part as if it is a text in itself. (i.e., 420 texts now!)
\end{itemize}
\end{frame}

\begin{frame}
\frametitle{... continued}
\begin{itemize}
\item So, each of these 420 texts is represented as word-frequency tables before proceeeding to next step. \pause
\item Once we have this, next step is to build a term document matrix \pause 
\item He does "bag of words" classification i.e., each word is a potential "signature feature". \pause
\item If we take all words, it will be too many and the "learning" becomes slow. So, he talks about removing words that don't appear frequently enough to be useful. 
\item Once all this is done, the the actual "learning" part is only one line of R code!
\item Once done, that "learned model" can be used for predicting the authorship of the unknown texts. 
\end{itemize} 
\end{frame}

\begin{frame}
\frametitle{Rest of Today's class}
\begin{itemize}
\item I created small functions to do all the corpus analyses we learnt so far, and stored them in a R file.
\item There is a small tutorial associated. It has exercises at the end.
\item Go through the tutorial - line by line - try to understand what is happening. Perhaps add comments to yourself. 
\item Once you understand whats going on, start doing the exercises.
\item We will primarily continue on this in the evening tutorial (if anyone comes)
\item Ask questions, and I suggest working in groups of 2 and discussing with your teammate.
\end{itemize} %TODO
\end{frame}

\begin{frame}
\frametitle{Next Week}
\begin{itemize}
\item movie reviews sentiment analysis using tm library. 
\item no readings. Just spend some time revising what we learnt so far - it will be useful.
\end{itemize}
\end{frame}

\end{document}
